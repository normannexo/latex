\documentclass[a4paper,11pt]{article}

\usepackage[ansinew]{inputenc}
\usepackage[T1]{fontenc}
\usepackage{lmodern}
\usepackage[ngerman]{babel}
\usepackage[automark]{scrpage2}

\usepackage{amsmath}
\usepackage{amssymb}
\usepackage{amsthm}

\usepackage[
  hypertexnames=false,
  pdftitle={Einsendeaufgaben Kurseinheit 1},
  pdfauthor={Torben Flickinger},
  bookmarksopen,
  colorlinks,
  linkcolor=blue,
  urlcolor=black%
]{hyperref}
\pagestyle{scrheadings}
\ihead{Einsendeaufgaben Kurseinheit 1}
\ohead{Torben Flickinger, Matrikelnr. 8278490 }
\cfoot{\pagemark}
\begin{document}
%Ab hier kommt ihr Text%


 \section*{Aufgabe 1.3}
 
 \begin{enumerate}
 \item
 Es gilt $n^2 > n + 1$ f�r alle $n\ge2.$\\
 Induktionsanfang: Sei $n_0=2$. Dann gilt $2^2>2+1\Rightarrow4>3$\\\\
 Induktionsannahme: F�r ein $n\ge2$ gilt $n^2>n+1$\\\\
 Induktionsschritt: Zu zeigen ist
 \[
 	n^2>n+1\Rightarrow(n+1)^2>(n+1)+1=n+2
 \]
 Es gilt
 \begin{eqnarray*}
 	(n+1)^2&=&n^2+2n+1>n+1+2n+1\\
 	&\ge&(n+1)+5\text{, denn }n\ge2\\
 	&>&n+2.
 \end{eqnarray*}
 Mit dem Prinzip der vollst�ndigen Induktion folgt, dass $n^2>n+1$ f�r alle $n\in\mathbb{N},n\ge2$ gilt.
\item
Es gilt $n^2 \ge 2n + 3$ f�r alle $n\ge3.$\\
 Induktionsanfang: Sei $n_0=3$. Dann gilt $3^2\ge6+3\Rightarrow9>9$\\\\
 Induktionsannahme: F�r ein $n\ge3$ gilt $n^2\ge2n+3$\\\\
 Induktionsschritt: Zu zeigen ist
 \[
 	n^2\ge2n+3\Rightarrow(n+1)^2\ge2(n+1)+3=2n+5
 \]
 Es gilt
 \begin{eqnarray*}
 	(n+1)^2&=&n^2+2n+1\ge2n+3+2n+1\\
 	&\ge&2n +3+7\text{, denn }n\ge3\\
 	&>&2n+5.
 \end{eqnarray*}
 Mit dem Prinzip der vollst�ndigen Induktion folgt, dass $n^2>2n+3$ f�r alle $n\in\mathbb{N},n\ge3$ gilt.
 \end{enumerate}
\section*{Aufgabe 1.4}
\begin{enumerate}
\item
zum Beispiel:

$X=\begin{pmatrix} -2 & - 4 &-6\\1 & 2 & 3\end{pmatrix}$

\item

$AB=\begin{pmatrix} -1 & -8 & -10 \\ 1 & -2 & -5 \\ 9 & 22 & 15\end{pmatrix}$\\
$BA=\begin{pmatrix} 15 & -21  \\ 10 & -3 \end{pmatrix}$



\end{enumerate}

\section*{Aufgabe 1.5}
\begin{enumerate}
\item
Sei $f$ definierten durch $f(n)=1$ f�r alle $n\in\mathbb{N}$, wenn $n$ ungerade, und $f(n)=\frac{n}{2}$ f�r alle $n\in\mathbb{N}$, wenn $n$ gerade. Dann ist $f$ surjektiv, da ganz $\mathbb{N}$ in der Bildmenge enthalten ist, und 1 wird unendlich oft getroffen, da alle ungeraden Zahlen auf 1 abgebildet werden.
\item
Sei $f:\mathbb{N}\rightarrow\mathbb{N}$ definiert durch $f(n)=2n$. Dann ist $f$ injektiv, da jedes Element Bild von f nur von einem Element des Urbildes getroffen wird. Und die unendlich vielen ungeraden Zahlen sind nicht im Bild von $f$.
	

\end{enumerate}

\section*{Aufgabe 1.6}
Sei $A\in M_{mn}(\mathbb{K})$ eine Matrix, sodass $X A = 0\in M_{mn}(\mathbb{K})$ f�r alle Matrizen $X\in M_{mm}(\mathbb{K})$. Dann gilt insbesondere: $E_{ii}A = \begin{pmatrix}
0 & \cdots & 0 \\
0 & \cdots & 0 \\
\end{pmatrix}
$

%Dies letzte Zeile muss unbedingt erhalten bleiben%
%Hiernach also nicht mehr �ndern%
\end{document}